\documentclass[polish]{article}

\usepackage[polish]{babel}

\usepackage[utf8]{inputenc}
\usepackage[T1]{fontenc}

\usepackage{fancyvrb}
\usepackage{varioref}
\usepackage{tocloft}
\usepackage{float}

\DefineVerbatimEnvironment{codeblock}{Verbatim}{%
resetmargins=true,numbers=left,numbersep=6pt,xleftmargin=2em,fontsize=\footnotesize}
\floatstyle{ruled}
\newfloat{Program}{tbp}{lst}

\begin{document}

\section{Implementacje algorytmów}

\noindent\ldots algorytm ten implementuje procedura w Javie zawarta w programie~\vref{listing:code1}. 

\begin{Program}[p]
\begin{codeblock}
// Search for pattern phrases
BooleanQuery query = new BooleanQuery();
for (int j = 0; j < Math.min(100, featureVector.size()); j++) {
    final TermQuery tk = new TermQuery(
        new Term("keywords", featureVector.get(j).feature));
    tk.setBoost((float) fv.get(j).weight);
    query.add(tk, BooleanClause.Occur.SHOULD);
}
\end{codeblock}
\caption{Fragment kodu odpowiadający za obliczenia XXX.}\label{listing:code1}%
\end{Program}

\end{document}