
\chapter{Writing Your Thesis in \LaTeX}

\section{Simple Typography}

Each paragraph is a block of text delimited with at least one empty line. So, this
paragraph is separated from the next one with exactly a single line (although
more could be used).

And this one is the consecutive paragraph. Indentation is created automatically by
\LaTeX{}. If you really need to have a paragraph without an indent, use a special
\texttt{noindent} command at the beginning of the paragraph.

\noindent This paragraph starts with no indentation. It is recommended that you break
paragraph lines at some sensible column. If you use soft line breaking in your editor,
\LaTeX{} may have problems reading very long paragraphs.


\section{Citing and Referencing}

There are two types of citations: external citation that goes into the bibliography section
and internal reference to another chapter or section.

\subsection{Citing Books, Articles and Other Works}

Use the \texttt{cite} command to cite works of others. Citations should be used if you really cite
the text literally (put it in quotes), but also if you refer to intellectual achievements of other
people. For example, we all appreciate Donald Knuth's work on \TeX{}~\cite{Knuth:ct-a} system and Leslie
Lamport's macro package --- \LaTeX{}~\cite{Lamport:LDP85}.


\section{FCM template macros}

There are several macro definitions in ppfcmthesis style. They are defined to ensure
consistency in your thesis, but you should feel free to redefine them up to your taste.
See Table~\ref{tab:macros} for a list of macros and examples of their use.

\begin{table}[t]
\caption{Macro definitions defined in ppfcmthesis style and examples of their use.}\label{tab:macros}%
\begin{center}\footnotesize
\begin{tabular}{>{\raggedright}p{1.5cm} | p{5cm} | p{5cm}}
\tabhead{Macros} \vline& \tabhead{Example code} \vline& \tabhead{Final version} \\ \hline
    \texttt{termdef} \texttt{acronym} 
    & \texttt{we call this a $\backslash$termdef\{Database Management System\}
      ($\backslash$acronym\{DBMS\})}
    & [\ldots] we call this a \termdef{Database Management System} (\acronym{DBMS}) [\ldots] \\ \hline

    \texttt{definicja} \texttt{akronim} \texttt{english} 
    & \texttt{nazywamy go $\backslash$definicja\{systemem zarządzania bazą danych\}
    ($\backslash$akronim\{DBMS\}, $\backslash$english\{Database Management System\})}
    & [\ldots] nazywamy go \definicja{systemem zarządzania bazą danych} (\akronim{DBMS}, \english{Database Management System}) [\ldots] \\ \hline
\end{tabular}
\end{center}
\end{table}

