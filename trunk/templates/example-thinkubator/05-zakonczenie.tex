
\chapter{Podsumowanie}
\label{sec:Podsumowanie}

\section{Stan Systemu}
Projekt zakończył się sukcesem, ponieważ w~trakcie pisania pracy był już w~pełni
funkcjonalny -- zaimplementowano wszystkie określone w~wymaganiach funkcje.
Pokonano wiele trudności z~dziedziny informatyki, automatyki oraz termodynamiki.
Uzyskano jakość sterowania temperaturą dorównującą lub nawet przekraczającą tą
w~obecnie stosowanych komercyjnie rozwiązaniach.  System Thinkubator umożliwia
rozproszoną wymianę danych oraz zdalne ustawianie urządzeń.  Użytkownik może
korzystać z~przyjaznego interfejsu podczas wykonywania tych trudnych czynności.
W~najbliższym czasie twórcy będą się konsultować z~przedstawicielami
Poznańskiego Nowego Zoo w~celu wdrożenia systemu oraz przeprowadzenia pierwszego
praktycznego testu -- próby wyklucia, który ostatecznie sprawdzi funkcjonalność
Thinkubatora.

\section{Dalszy Rozwój}
Głównym celem systemu Thinkubator jest ułatwienie pracy osobom zajmującym się
sztuczną inkubacją ptaków. Jego celem drugorzędnym jest wsparcie dla osób
zajmujących się badaniem tego procesu. W~tej chwili stworzony system jest
platformą umożliwiającą implementację zajmujących się tym rozszerzeń. Po
wdrożeniu, gdy tabele bazy danych Centrum Nadzoru wypełnią się informacjami
z~przeprowadzonych inkubacji oraz ich wynikami, możliwe będzie opracowanie
i~wdrożenie zaawansowanych algorytmów wyszukiwania wzorcowych przebiegów
procesu. Do tego celu potrzebne będą także konsultacje z~osobami zajmującymi się
badaniami nad inkubacją od strony biologicznej. 

Inną opcją systemu jest dostosowanie go do wylęgu gadów, płazów, owadów,
pajęczaków, gdzie również konieczna jest precyzyjna dobowa manipulacja
temperaturą. Podczas tworzenia systemu Thinkubator nie brano pod uwagę tych grup
zwierząt, jednak ze względu na wielkie możliwości stworzonego systemu, można go
będzie łatwo dostosować do wymogów inkubacji wymienionych gromad.  Zastosowanie
wymiennika ciepła pozwala stworzyć urządzenie przystosowane do każdego rodzaju
jaj, ponieważ nie jest on integralną częścią komory inkubacyjnej.
 
\section{Wnioski}
Realizacja projektu wymagała od twórców wiedzy z~bardzo wielu dziedzin szeroko
pojętej inżynierii. Podczas zbierania trudnych do zrozumienia wymagań konieczne
było poznania ornitologicznego żargonu. Z~kolei zastosowane rozwiązania
programistyczne były dobrane tak, aby twórcy mieli styczność z~najnowszymi
i~najciekawszymi narzędziami. Praca ze sprzętem wymagała dostosowywania
koncepcji do dostępnych zasobów oraz często zmieniającej się sytuacji (np.
awarie), a~praca w~czteroosobowym zespole wymagała sporej wiedzy z~dziedziny
zarządzania (nie tylko projektem informatycznym). Dzięki temu autorzy projektu
zyskali najcenniejszą rzecz -- doświadczenie.  

