
\chapter{Wstęp}
Realizacja niniejszej pracy inżynierskiej została rozpoczęta w~grudniu 2005 roku
jako projekt w~konkursie
\akronim{\htmladdnormallink{CSIDC}{http://computer.org/csidc}}
(\english{Computer Society International Design Competition}). Organizatorem
konkursu jest Towarzystwo Komputerowe przy amerykańskim Instytucie Inżynierii
Elektrotechnicznej i~Elektronicznej (\english{IEEE Computer Society}). Konkurs
ten zachęca studentów do pracy w~zespole w~celu stworzenia opartego na
wykorzystaniu komputerów rozwiązania dla wybranego problemu. Zadaniem studentów
jest zaprojektowanie, wykonanie, przetestowanie, udokumentowanie, a~nawet
sprzedanie wynalezionego przez nich systemu. Konkurs rozgrywany jest w~trzech
etapach. W~pierwszym etapie studenci analizują zadany temat, wybierają problem i
zgłaszają temat pracy. Ze wszystkich zgłoszeń wybieranych jest 300 drużyn, które
przystępują do zgłębienia problemu i~przygotowują plan realizacji projektu.
Rezultatem jest raport wstępny (\english{Interim Report}), na podstawie którego
100 najlepszych drużyn kwalifikuje się do trzeciego etapu. Następnie zespoły te
realizują zaplanowane prace i~przygotowują raport finałowy (\english{Final
Report}). Autorzy dziesięciu najlepszych prac są zapraszani na finały do
Waszyngtonu, D.C., gdzie zespoły prezentują swoje projekty. Na podstawie
prezentacji i~raportu finałowego wybierani są zwycięzcy konkursu. Temat
\akronim{CSIDC} w~2006 roku brzmiał: ,,Preserving, Protecting and Enhancing the
Environment'', czyli w~wolnym tłumaczeniu ,,Podtrzymywanie, Ochrona i~Ulepszanie
Środowiska''. Niniejszy projekt został zakwalifikowany do III etapu konkursu,
a~następnie był rozwijany i~doczekał się obecnej formy pracy inżynierskiej.

Szeroko pojęta ochrona środowiska kojarzona jest zazwyczaj z~zanieczyszczeniami
środowiska (powietrza, wody lub gleby) czy globalnym ociepleniem. Do kwestii
wymierania gatunków przywiązuje się wagę dużo mniejszą, podczas gdy niezmiernie
istotne jest to, że spośród wielu codziennie wymierających gatunków każdy
ewoluował do obecnej postaci miliony lat i~ma wielki wpływ na równowagę swojego
ekosystemu. Pomoc w~przetrwaniu chociaż jednego gatunku wnosi istotny wkład
w~ochronę środowiska.

Inspiracją do napisania niniejszej pracy były rozmowy z~pracownikami Poznańskiego
Nowego Zoo. Dyrektor do spraw hodowli tego ogrodu, dr~inż.~Radosław Ratajszczak,
opowiedział o~następujących wydarzeniach:
\begin{quote}
	 W~1974 roku światowa populacja jednej z~odmian pustułki (łac. \emph{mauritius
	 kestrel}) liczyła 10~sztuk, wśród których były tylko 2~samice. Wszystkie
	 ptaki żyły w~niewoli -- gatunek był bardzo bliski wymarcia. Minęły dwa lata
	 zanim jedna z~samic złożyła jedno jajo. Zostało ono wzięte do inkubatora, by
	 zapewnić mu przetrwanie. Inkubacja się udała, pisklę się wykluło, ale
	 wydarzył się straszny wypadek: eter, który był używany w~czujniku inkubatora
	 wyciekł i~zatruł pisklę. Jak widać, bezsensowny wypadek o~mały włos nie
	 doprowadził do wyginięcia gatunku.
\end{quote}
Na szczęście pustułki nie wymarły, a~ich dzisiejsza populacja oceniana jest
na~ok.~150 sztuk. Mimo to ta historia powinna być traktowana jako
ostrzeżenie, bo podobny wypadek w~przyszłości może mieć fatalne konsekwencje.

Inkubacja zagrożonych gatunków jest bardzo trudnym zadaniem i~wymaga
odpowiednich narzędzi. Analiza problemu wykazała że~istniejące inkubatory
zapewniają tylko podstawową funkcjonalność, niewystarczającą dla wielu
gatunków, a~także że brakuje inkubatorów ułatwiających prowadzenie badań
naukowych. Podstawowa funkcjonalność której oczekują ornitolodzy to możliwość
analizy przebiegu inkubacji (zapisywanie stanu inkubatora i~możliwość jego
wizualizacji) czy też programowalność (możliwość zaprogramowania zmiany
parametrów inkubacji w~czasie). Dodatkowo pożądana była możliwość zbierania
i~analizowania danych z~wielu inkubatorów.

Powyższe przemyślenia zainspirowały autorów niniejszej pracy do stworzenia
Thinkubatora -- systemu umożliwiającego inkubację ptasich jaj w~kontrolowanym
i~nadzorowanym środowisku, który może zrewolucjonizować podejście do inkubacji.
W~centrum systemu znajduje się wysokiej jakości inkubator. Jest on wyposażony
w~komputer przemysłowy z~interfejsem ethernetowym umożliwiającym komunikację ze
Stacją Kontrolną -- aplikacją zainstalowaną na~komputerze klasy PC umożliwiającą
bezpośrednią interakcję z~inkubatorami. Za pomocą Stacji Kontrolnej można
programować i~nadzorować wiele inkubatorów -- oszczędza się dzięki temu czas
potrzebny na programowanie wielu urządzeń. Stacja Kontrolna umożliwia także
zapisywanie informacji o~stanie inkubacji w~celu ich wizualizacji lub do
późniejszej analizy. Cały system jest zintegrowany ze zdalnym serwerem --
Centrum Nadzoru. Serwer zbiera dane z~wszystkich inkubatorów umożliwiając
zaawansowaną analizę statystyczną lub analizę przy pomocy algorytmów wspomagania
decyzji. Umożliwia on także wymianę informacji pomiędzy użytkownikami na całym
świecie. Posiada również funkcję wizualizacji danych, która pozwala na
współdzielenie wiedzy o~ptakach pomiędzy profesjonalistami a~hobbistami i~jej
poszerzanie.

Ornitolodzy z~Zoo w~Poznaniu uważają, że system o~takich parametrach jest
wysoce pożądany. Analiza kosztów, zalet i~wad tego systemu pokazała, że mógłby
on zostać bez problemu wdrożony w~wielu ogrodach zoologicznych na świecie.

Pomimo dużej funkcjonalności i~zastosowania wielu przydatnych rozwiązań koszt
urządzenia wraz z~wdrożeniem byłby niższy niż koszt wdrożenia zwykłego
inkubatora o~podobnych parametrach.

W realizacji projektu brały udział cztery osoby. Tabela \ref{tab:Podzial}
przedstawia zadania wykonane przez każdą z nich.

Struktura pracy jest następująca. W~rozdziale~\ref{sec:CeliZakres} opisano
obecny stan wiedzy na temat inkubacji ptasich jaj, stosowane rozwiązania
w~komercyjnych inkubatorach oraz dokładnie zdefiniowano cel projektu.
W~rozdziale~\ref{sec:Architektura} opisano architekturę systemu Thinkubator.
Jest on podzielony na cztery podrozdziały, opisujące realizację poszczególnych
elementów systemu oraz zastosowane metody komunikacji między nimi.
W~rozdziale~\ref{sec:Praktyka} opisano przykładowy scenariusz użycia systemu
wraz z~krókim opisem interfejsu użytkownika. Praca zakończona jest krótkim
podsumowaniem w~rozdziale~\ref{sec:Podsumowanie}.

\begin{table}[b]
	\centering
	\begin{tabular}{|m{6cm}|c|c|c|c|}\hline
		\bfseries Czynność & \bfseries Janusz & \bfseries Paweł & \bfseries Tomasz &
		\bfseries Szymon \\
		& \bfseries Bossy & \bfseries Lubarski & \bfseries Nowak & \bfseries
		Szafraniec \\\hline
		\multicolumn{5}{|c|}{\bfseries Budowa inkubatora} \\\hline
		Wysokopoziomowy projekt inkubatora & $\checkmark$ & $\checkmark$ &
		$\checkmark$ & $\checkmark$ \\\hline
		Projekt wymiennika ciepła & $\checkmark$ & $\checkmark$ & $\checkmark$ &
		$\checkmark$ \\\hline
		Wykonanie wymiennika ciepła & $\checkmark$ & $\checkmark$ & & $\checkmark$
		\\\hline
		Wykonanie konstrukcji zewnętrznej & $\checkmark$ & $\checkmark$ & &
		$\checkmark$ \\\hline
		Projekt komory inkubacyjnej & $\checkmark$ & $\checkmark$ & & $\checkmark$
		\\\hline
		Montaż elektroniki & $\checkmark$ & $\checkmark$ & $\checkmark$ &
		$\checkmark$ \\\hline 
		Wykonanie elektroniki peryferyjnej & & & $\checkmark$ & \\\hline
		Montaż inkubatora & $\checkmark$ & $\checkmark$ & $\checkmark$ &
		$\checkmark$ \\\hline 
		\multicolumn{5}{|c|}{\bfseries Oprogramowanie inkubatora} \\\hline
		Projekt wysokopoziomowy & $\checkmark$ & $\checkmark$ & $\checkmark$ &
		$\checkmark$ \\\hline 
		Implementacja jądra systemu & $\checkmark$ & & & \\\hline
		Projekt algorytmów sterowania & $\checkmark$ & $\checkmark$ & & $\checkmark$
		\\\hline
		Implementacja algorytmów sterowania & $\checkmark$ & $\checkmark$ & &
		$\checkmark$ \\\hline
		Kalibracja algorytmów sterowania & $\checkmark$ & $\checkmark$ & &
		$\checkmark$ \\\hline
		Oprogramowanie UI & & & $\checkmark$ & \\\hline
		Uruchomienie \emph{SBC} & $\checkmark$ & $\checkmark$ & $\checkmark$ &
		$\checkmark$ \\\hline
		Implementacja algorytmów komunikacji & $\checkmark$ & $\checkmark$ & & $\checkmark$
		\\\hline
		\multicolumn{5}{|c|}{\bfseries Stacja Kontrolna} \\\hline
		Projekt wysokopoziomowy & $\checkmark$ & $\checkmark$ & & $\checkmark$
		\\\hline
		Implementacja & $\checkmark$ & & & $\checkmark$ \\\hline
		Implementacja algorytmów komunikacji & $\checkmark$ & & & $\checkmark$
		\\\hline
		\multicolumn{5}{|c|}{\bfseries Centrum Nadzoru} \\\hline
		Projekt wysokopoziomowy &		$\checkmark$ & $\checkmark$ & & $\checkmark$
		\\\hline
		Implementacja UI & & $\checkmark$ & & \\\hline
		Projekt schematu bazy danych & $\checkmark$ & $\checkmark$ & & $\checkmark$
		\\\hline
		Implementacja algorytmów komunikacji & & $\checkmark$ & & \\\hline
	\end{tabular}
	\caption{Podział prac}
	\label{tab:Podzial}
\end{table}
