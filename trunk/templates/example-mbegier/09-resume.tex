\chapter{Wnioski i podsumowanie prac}
Rozdział ten stanowi krótkie podsumowanie prac związanych z budową benchmarku,
jak i odpowiedź na postawione na początku pracy pytania i cele. Wskazuje także kierunki dalszego rozwoju.

\section{Wyniki i wnioski z prac}
Podstawowym celem niniejszej pracy było stworzenie benchmarku, który umożliwiałby mierzenie
wydajności SZBD. Autorowi nie chodziło jednak o stworzenie typowego benchmarku
wraz ze związanymi z nim ograniczeniami min.:
\begin{itemize}
\item jednym niemodyfikowalnym modelem,
\item występowaniem często jako specyfikacja, którą trzeba dla każdego DBMS niezależnie implementować,
\item testowaniem jednego konkretnego DBMS.
\end{itemize}
Głównym celem było przezwyciężenie powyższych ograniczeń i stworzenie narzędzia,
bardziej uniwersalnego, posiadającego szerszą gamę zastosowań.
Co zatem uzyskano? Czy stworzony benchmark spełnia powyższe kryteria?
W stworzonym benchmarku definicja testu, procedury testowej, składa się z trzech modeli.
Podział taki, ułatwia definiowanie testu, każdy model ma bowiem ściśle określone przeznaczenie, 
razem zaś stanowią pełną definicję, uwzględniającą wszystkie aspekty rzeczywistego 
systemu opartego o współpracę z DBMS. Takie podejście ułatwia definiowanie testu niezależnie od DBMS. Zmiana DBMS 
sprowadza się do zmiany dialektu i parametrów połączenia w modelu testu. Koncepcja 
taka zapewnia również łatwą, wielokryterialną skalowalność. Sama procedura testu jest niezależna 
od jednego konkretnego modelu bazy danych, czy modelu obciążenia. Wykorzystanie 
języka Java i sterowników JDBC umożliwia testowanie różnych DBMS, jak i uruchamianie
benchmarku na różnych systemach operacyjnych.

Zbudowany benchmark charakteryzują najpełniej następujące cechy:
\begin{itemize}
\item niezależność modelu od konkretnego DBMS,
\item niezależność procedury testu od konkretnego modelu,
\item niezależność mechanizmów analizy wyników od konkretnego modelu,
\item możliwość testowania różnych DBMS (Oracle, PostgreSQL, MySQL),
\item możliwość testowania różnych baz danych,
\item możliwość definiowania własnych modeli baz danych i obciążenia,
\item możliwość analizy modelu bazy danych rzeczywistego systemu i analizy modelu obciążenia,
\item możliwość porównywania wydajności różnych sterowników JDBC.
\end{itemize}

\section{Propozycje dalszego rozwoju}

Budując dowolny system informatyczny w pierwszej kolejności należy wyrzec się 
przeświadczenia o własnej nieomylności, a także przekonania, że stworzone rozwiązanie
będzie rozwiązaniem idealnym, nie wymagającym w przyszłości zmian. 
Użytkownicy są bowiem najlepszym weryfikatorem stworzonego rozwiązania i podjętych decyzji projektowych.
To bowiem zbudowane rozwiązanie jest dla nich i nigdy na odwrót.
W świecie inżynierii oprogramowania istnieje powiedzenie, iż ,,system nie jest dobry
przed wersją trzecią''. Trudno nie zgodzić się z powyższymi stwierdzeniami. Również
stworzony benchmark, jest pewnym kompromisem. 

Autor zatem, świadomy opisanych powyżej problemów dołożył wszelkich starań by 
zbudowane rozwiązanie było łatwe w rozbudowie czy modyfikacji.

Jeżeli benchmark znajdzie zastosowanie to z pewnością w przyszłości będzie istniała konieczność
wprowadzania w nim modyfikacji. Już na etapie prac zauważono, iż dla modeli
bazy danych i obciążenia najlepszym rozwiązaniem było by stworzenie edytorów graficznych.
Ze względu jednak na inny priorytet zadań i celów pracy, stworzenie takich edytorów 
zostało wyłączone z niniejszej wersji benchmarku. Nie mniej jednak postanowiono
wykorzystać rozwiązania ułatwiające zbudowanie takich edytorów w przyszłości.
W procesie budowy aplikacji graficznej serwera benchmarku użyto zatem środowiska
Eclipse RCP~\cite{RCP1} w wersji 3.2.

Benchmark można by również rozszerzyć o testy w oparciu o scenariusze. Scenariusz określa 
z ustalonym prawdopodobieństwem kolejność wykonywania operacji/transakcji po sobie. Umożliwia zatem
stworzenie testów jeszcze bardziej zbliżonych do rzeczywistości.


