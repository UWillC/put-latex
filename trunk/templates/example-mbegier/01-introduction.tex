
\chapter{Wstęp}

Rozdział ten ma przybliżyć czytelnikowi motywacje jak i zakres realizowanej pracy magisterskiej, 
zawiera również krótki opis planu realizacji prac.

\section{Motywacje}\label{sect:Motywacje}

Obecnie na rynku dostępnych jest kilka benchmarków do testowania wydajności \akronim{DBMS}~\cite{DBMS}, 
różnią się one pomiędzy sobą zarówno zakresem testowanych właściwości, systemami dla których
są przeznaczone np.: \akronim{OLTP} (\english{Online Transaction Processing}~\cite{OLTP}) 
czy \akronim{OLAP} (\english{On Line Analytical Processing}~\cite{OLAP}),
jak i obsługiwanymi DBMS. Posiadają jednak z reguły pewne istotne ograniczenia:
\begin{itemize}
\item związane są z jednym niemodyfikowalnym modelem bazy danych i obciążenia,
\item występują często jako specyfikacja -- tj.~wymagają implementacji specyfikacji benchmarku niezależnie dla każdego DBMS,
\item umożliwiają testowanie jednego konkretnego DBMS.
\end{itemize}

Czy istnieje możliwość stworzenia benchmarku na tyle uniwersalnego,
by możliwe było definiowanie dla niego różnych modeli nie tylko baz danych,
ale również modeli obciążenia? Ponadto by modele te, zapisane w sposób niezależny
od konkretnego DBMS, mogły być wykorzystywane przez benchmark do testowania różnych DBMS?
Czy możliwe jest stworzenie benchmarku nie tylko przeznaczonego do porównywania wydajności
różnych DBMS, ale również do szukania najlepszego modelu bazy danych dla tworzonego rozwiązania, systemu?
Odpowiedzią na powyższe pytania ma być niniejszy benchmark i związana z nim praca magisterska.

\section{Cel i zakres pracy}

Celem pracy jest stworzenie oprogramowania (benchmarku) do testowania wydajności
\definicja{systemów zarządzania bazą danych} (\akronim{DBMS}, \english{Database Management System}). 
Benchmark powinien symulować rzeczywisty model bazy danych np.: sklep internetowy, bank, lub hurtownię itp.
Model obciążenia powinien się skalować, tzn.~umożliwiać zwiększanie rozmiaru w miarę wzrostu prędkości komputera.
Należy przeprowadzić testy prędkości dla operacji języka \definicja{SQL} (\english{Structured Query Language}~\cite{SQL}) typu 
\code{select}, \code{insert}, \code{update}, \code{delete}.
Testy powinny obejmować zależności prędkości od rozmiarów bazy danych i od liczby klientów (jednocześnie podłączonych). 
Wyniki trzeba wszechstronnie opracować w postaci tekstowej i graficznej (średnie, mody, mediany, histogramy).
Językiem programowania będzie Java, do łączenia z bazami danych użyty zostanie interfejs \akronim{JDBC} 
(\english{Java Database Connectivity}~\cite{JDBC1}).

Praca nad projektem została podzielona wstępnie na kilka etapów. Poniżej zamieszczono krótki ich opis.

\paragraph{Zapoznanie się z istniejącymi rozwiązaniami}
Tworzenie oprogramowania do testowania wydajności różnego typu elementów systemów komputerowych,
w tym systemów zarządzania bazami danych, nie jest zagadnieniem nowym, dlatego przed rozpoczęciem prac, 
należy zapoznać się z istniejącymi rozwiązaniami, by ,,nie odkrywać Ameryki na nowo''.
W tym celu przewidziano zapoznanie się z produktami organizacji \akronim{TPC} (\english{Transaction Processing Performance Council}) 
-- \url{http://www.tpc.org }, a w szczególności ze specyfikacją benchmarku 
TPC-C dostępną pod adresem \url{http://www.tpc.org/tpcc}.

\paragraph{Stworzenie modelu rozwiązania}
W tym etapie głównym celem będzie stworzenie modelu benchmarku tak, by model ten
spełniał zarówno cele pracy, jak i był odpowiedzią na pytania postawione w 
podrozdziale~\ref{sect:Motywacje}.

\paragraph{Implementacja benchmarku na niskim poziomie}
W pierwszej kolejności zostanie stworzona struktura benchmarku: struktura komunikacyjna
dla zapewnienia wymiany danych pomiędzy serwerem, a klientami \akronim{RTE} (\english{Remote Terminal Emulator}).
Wytworzone zostaną także elementy związane z generacją populacji bazy danych, 
analizą modeli, przygotowywaniem skryptów testowych, oraz ich wykonywaniem na RTE.

\paragraph{Testowanie struktury różnych DBMS}
Struktura testu zostanie przetestowana na kilku systemach baz danych: 
Oracle 10g~\cite{Oracle1}, PostgresSQL 8.1~\cite{PostgreSql1}, MySQL 5.0~\cite{MySql1}.

\paragraph{GUI serwera}
W tym kroku zostanie stworzone \akronim{GUI} (\english{Graphical User Interface}~\cite{GUI}) dla serwera benchmarku,
dla klientów RTE -- interfejs pozostanie tekstowy. Interfejs ten będzie umożliwiał
tworzenie modeli przy pomocy formularzy, zarządzanie klientami RTE, oraz przeprowadzanie procedury testu. 

\paragraph{Generator raportów}
Ostatnim etapem prac będzie dodanie do systemu generatora raportów, w postaci tekstowej
i graficznej. Dodane zostaną histogramy obrazujące rozkład czasów odpowiedzi dla
operacji i transakcji.
