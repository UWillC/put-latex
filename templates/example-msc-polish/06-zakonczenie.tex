\chapter{Podsumowanie i kierunki dalszego rozwoju}

Nadrzędnym celem niniejszej pracy było napisanie (w formie biblioteki) implementacji wybranych
algorytmów tworzenia tablic sufiksów w celu analizy ich efektywności i zachowania w języku wysokiego
poziomu, jakim jest język \texttt{Java}. Zadanie to obejmowało zapoznanie się z~dostępną literaturą
poświęconą tematyce tablic sufiksów, wybór najciekawszych algorytmów oraz ich opisanie. Dodatkowo,
w~wyniku pracy powinna była powstać klasyfikacja algorytmów tworzenia tablic sufiksów.

Największym wyzwaniem podczas tworzenia pracy było znalezienie najlepszych spośród algorytmów
tworzenia tablic sufiksów. Bardzo pomocne były w~tym opracowania zawierające zestawienia algorytmów
\cite{taxonomy, schurmann-phd} oraz publikowane w~internecie wyniki testów wydajnościowych
[\ref{msufsort}, \ref{mori-benchmark}]. Pewnym problemem podczas implementacji algorytmów było tłumaczenie
do języka \texttt{Java} takich konstrukcji języka \texttt{C++}, jak instrukcje \texttt{define},
wskaźniki oraz inne polecenia preprocesora.

Podsumowując, należy stwierdzić, że wszystkie zakładane cele zostały zrealizowane. Przegląd
literatury, opisy algorytmów oraz ich klasyfikacja znalazły się w~tekście tej pracy, podobnie jak
wyniki testów wydajnościowych powstałej implementacji. Według naszej wiedzy niniejsza praca jest
pierwszą publikacją w~języku polskim poświęconą w całości tematyce algorytmów tworzenia tablic
sufiksów. Zaimplementowana biblioteka programowa jest zaś na dzień dzisiejszy jedyną taką pozycją
dedykowaną algorytmom tworzenia tablic sufiksów napisaną w~języku \texttt{Java}. Kod źródłowy
biblioteki jest udostępniony na licencji BSD, można go pobrać ze strony projektu:
\url{http://www.jsuffixarrays.org}.

Przewidywany dalszy rozwój biblioteki zakłada opracowanie algorytmu dobierającego najlepszą metodę
tworzenia tablic sufiksów na podstawie charakterystyki danych wejściowych (rozmiaru alfabetu i
rozkładu symboli w danych wejściowych). Interesującym kierunkiem rozwoju biblioteki jest również
stworzenie własnej implementacji algorytmu \emph{improved two-stage}, dostosowanej do specyfiki
języka \texttt{Java}.
